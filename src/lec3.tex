\section{Lec 3}

\subsection{Quadratic Functions}

\subsubsection{The Minimizer}

Given $\bm H\in \Sbb^n$, $\bm g\in \R^n$, $d\in \R$, consider
\[
f(\bm x)
= \frac12 \bm x^{\top}\bm H\bm x
+ \bm g^{\top}\bm x
+ d.
\]

\begin{theorembox}
The function $f$ is bounded below and has a minimizer if and only if:
\begin{enumerate}[label=\alph*)]
  \item $\bm H$ is positive semidefinite, and
  \item $\bm g \in \operatorname{Range}(\bm H)$.
\end{enumerate}
In this case, any solution of
\[
\bm H\bm x = -\bm g
\]
is a minimizer.
\end{theorembox}

\begin{factbox}
\textbf{Contrapositive Theorem.}  
If $\bm H$ is not positive semidefinite, or $\bm H$ is positive semidefinite and
$\bm g\notin \operatorname{Range}(\bm H)$, then $f$ is unbounded below.
\end{factbox}

\subsubsection{Proof of the Contrapositive}

\begin{proofbox}
\textbf{Case 1: (a) fails: $\bm H$ is not positive semidefinite.}

Then there exists $\bm z\in \R^n$ such that
\[
\bm z^{\top}\bm H\bm z < 0.
\]
Define the univariate function
\[
q(t) = f(t\bm z)
= \frac12 t^2 \bm z^{\top}\bm H\bm z + t\,\bm g^{\top}\bm z + d.
\]
This is a quadratic function in $t$ with \emph{negative} leading coefficient
$\frac12 \bm z^{\top}\bm H\bm z < 0$, so $q(t)$ attains arbitrarily negative values as $t\to \pm\infty$. Therefore $f$ has no lower bound and has no minimizer.

\bigskip

\textbf{Case 2: (a) holds but (b) fails: $\bm H$ is positive semidefinite and $\bm g\notin \operatorname{Range}(\bm H)$.}

By the Fundamental Theorem of Linear Algebra for symmetric $\bm H$,
\[
\R^n = \operatorname{Range}(\bm H) \oplus \operatorname{Null}(\bm H),
\]
with $\operatorname{Range}(\bm H) \perp \operatorname{Null}(\bm H)$. Thus there exist
\[
\bm g_1 \in \operatorname{Range}(\bm H),\quad
\bm g_2 \in \operatorname{Null}(\bm H)
\]
such that
\[
\bm g = \bm g_1 + \bm g_2,
\quad
\bm g_1^{\top}\bm g_2 = 0.
\]

Define
\[
q(t) := f(t\bm g_2).
\]
Since $\bm H\bm g_2 = \0$ (because $\bm g_2\in \operatorname{Null}(\bm H)$), we have
\[
\begin{aligned}
q(t)
&= \frac12 t^2 \bm g_2^{\top}\bm H\bm g_2
   + t\,\bm g^{\top}\bm g_2 + d \\
&= t\,\bm g^{\top}\bm g_2 + d.
\end{aligned}
\]
Next,
\[
\bm g^{\top}\bm g_2
= (\bm g_1 + \bm g_2)^{\top}\bm g_2
= \bm g_2^{\top}\bm g_2 > 0,
\]
because $\bm g\notin \operatorname{Range}(\bm H)\Rightarrow\bm g\neq \bm g_1\Rightarrow \bm g_2\neq \0$.

Hence $q(t)$ is a linear function of $t$ with positive slope. Therefore
\[
q(t)\to -\infty \quad\text{as}\quad t\to -\infty,
\]
so $f$ is unbounded below.
\end{proofbox}

\subsection{Linear Least Squares}

\subsubsection{Norms}

For $\bm x\in \R^n$, the Euclidean norm (or $2$-norm) is
\[
\|\bm x\| = \sqrt{x(1)^2 + \cdots + x(n)^2}.
\]

A \textbf{norm} $\|\cdot\|$ on $\R^n$ satisfies, for all $\bm x,\bm y\in \R^n$ and all $\lambda\in \R$:
\begin{enumerate}
  \item Positivity:
  \[
  \|\bm x\| > 0 \text{ for } \bm x\neq \0,
  \quad\text{and}\quad
  \|\bm x\| = 0 \Rightarrow \bm x = \0.
  \]
  \item Positive 1-homogeneity:
  \[
  \|\lambda \bm x\| = |\lambda|\cdot \|\bm x\|.
  \]
  \item Triangle inequality:
  \[
  \|\bm x + \bm y\|
  \le \|\bm x\| + \|\bm y\|.
  \]
\end{enumerate}

\subsubsection{Linear Least Squares (LLS)}

Given $\bm A\in \R^{m\times n}$ and $\bm y\in \R^m$, we want $\bm x\in \R^n$ to solve
\[
\min_{\bm x}\; \|\bm A\bm x - \bm y\|.
\]
This is equivalent to
\[
\min_{\bm x}\; \|\bm A\bm x - \bm y\|^2
\quad\text{or}\quad
\min_{\bm x}\; \frac12 \|\bm A\bm x - \bm y\|^2.
\]

Compute:
\[
\begin{aligned}
\frac12 \|\bm A\bm x - \bm y\|^2
&= \frac12 (\bm A\bm x - \bm y)^{\top}(\bm A\bm x - \bm y)\\
&= \frac12\bigl(\bm x^{\top}\bm A^{\top} - \bm y^{\top}\bigr)
   (\bm A\bm x - \bm y)\\
&= \frac12 \bm x^{\top}\bm A^{\top}\bm A\bm x
   - \frac12 \bm y^{\top}\bm A\bm x
   - \frac12 \bm x^{\top}\bm A^{\top}\bm y
   + \frac12 \bm y^{\top}\bm y\\
&= \frac12 \bm x^{\top}\bm A^{\top}\bm A\bm x
   - \bm y^{\top}\bm A\bm x
   + \frac12 \bm y^{\top}\bm y,
\end{aligned}
\]
since $\bm x^{\top}\bm A^{\top}\bm y = \bm y^{\top}\bm A\bm x$ (both scalars).

This is a quadratic function in $\bm x$ with
\[
\bm H := \bm A^{\top}\bm A,\quad
\bm g := -\bm A^{\top}\bm y,\quad
d := \frac12 \bm y^{\top}\bm y.
\]

\begin{theorembox}
For all $\bm A\in \R^{m\times n}$, the matrix $\bm A^{\top}\bm A$ is positive semidefinite.
\end{theorembox}

\begin{proofbox}
\[
\bm x^{\top}\bm A^{\top}\bm A\bm x
= (\bm A\bm x)^{\top}\bm A\bm x
= \|\bm A\bm x\|^2 \ge 0,\quad \forall \bm x\in \R^n.
\]
\end{proofbox}

\begin{theorembox}
For all $\bm A\in \R^{m\times n}$ and all $\bm y\in \R^m$,
\[
\bm A^{\top}\bm y \in \operatorname{Range}(\bm A^{\top}\bm A).
\]
\end{theorembox}

\begin{proofbox}
\textbf{(Idea.)}  
Otherwise, by the quadratic theorem for
\[
\frac12\|\bm A\bm x - \bm y\|^2
= \frac12 \bm x^{\top}\bm A^{\top}\bm A\bm x
  - \bm y^{\top}\bm A\bm x
  + \frac12 \bm y^{\top}\bm y,
\]
the objective would be unbounded below, contradicting the fact that squared norms are always nonnegative. Therefore a minimizer exists and must satisfy
\[
\bm A^{\top}\bm A\bm x = \bm A^{\top}\bm y.
\]
\end{proofbox}

This system is called the \textbf{system of normal equations}. The solution is unique if $\bm A^{\top}\bm A$ is positive definite.

\begin{theorembox}
For $\bm A\in \R^{m\times n}$, the following are equivalent:
\[
\bm A^{\top}\bm A \text{ is positive definite}
\;\Longleftrightarrow\;
\bm A^{\top}\bm A \text{ is nonsingular}
\;\Longleftrightarrow\;
\operatorname{rank}(\bm A) = n.
\]
\end{theorembox}

\begin{proofbox}
\textbf{(Sketch.)}  
If $\operatorname{rank}(\bm A)=n$, the columns of $\bm A$ are independent, so
\[
\bm A\bm x \neq \0 \quad \forall \bm x\neq \0.
\]
Hence
\[
\bm x^{\top}\bm A^{\top}\bm A\bm x
= \|\bm A\bm x\|^2 > 0 \quad \forall \bm x\neq \0,
\]
so $\bm A^{\top}\bm A$ is positive definite and thus nonsingular. The other implications follow from standard linear algebra facts about positive definite and nonsingular matrices.
\end{proofbox}

\subsection{Orthogonal Matrices}

Recall: if $\bm x^{\top}\bm y=0$, we say $\bm x$ and $\bm y$ are \textbf{orthogonal}.

If $\{\bm x_1,\dots,\bm x_k\}$ is an orthonormal set with
\[
\bm x_i^{\top}\bm x_j = \delta_{ij},
\]
and $\bm U$ is the matrix having these vectors as columns, then
\[
\bm U^{\top}\bm U = \I,\quad \operatorname{rank}(\bm U)=k\le n.
\]

If $\bm Q\in \R^{n\times n}$ is square and satisfies
\[
\bm Q^{\top}\bm Q = \I,
\]
then $\bm Q$ is an \textbf{orthogonal matrix}. In this case,
\[
\bm Q^{-1} = \bm Q^{\top},
\]
and the rows of $\bm Q$ are also orthonormal.

\begin{factbox}
\begin{itemize}
  \item If $\bm Q$ is orthogonal, then $\bm Q^{\top}$ is also orthogonal.
  \item If $\bm Q_1, \bm Q_2$ are orthogonal, then $\bm Q_1\bm Q_2$ is orthogonal as well.  
        (The set
        \[
        O(n) = \{\bm Q\in \R^{n\times n}: \bm Q^{\top}\bm Q = \I\}
        \]
        is a \emph{Lie group}.)
  \item For any orthogonal $\bm Q$ and any $\bm x\in \R^n$,
        \[
        \|\bm Q\bm x\| = \|\bm x\|.
        \]
\end{itemize}
\end{factbox}

Orthogonal matrices represent rotations or reflections.

\newpage

\subsection{Eigendecomposition}

\subsubsection{Spectral Theorem}

\begin{theorembox}
For any $\bm A\in \Sbb^n$, there exists an orthogonal matrix $\bm Q$ and a diagonal matrix $\bm D$ such that
\[
\bm A = \bm Q \bm D \bm Q^{\top}.
\]
The diagonal entries of $\bm D$ are the eigenvalues of $\bm A$ (unique up to ordering), and the columns of $\bm Q$ are the corresponding eigenvectors.

Equivalently,
\[
\bm A = \bm Q \bm D \bm Q^{\top} \Longleftrightarrow \bm A \bm Q = \bm Q \bm D.
\]
\end{theorembox}

\subsubsection{Characterizations of Positive Semidefiniteness}

\begin{theorembox}
For $\bm A\in \Sbb^n$, the following are equivalent:
\begin{enumerate}
  \item $\bm A$ is positive semidefinite.
  \item All eigenvalues of $\bm A$ are $\ge 0$.
  \item There exists a matrix $\bm G$ such that $\bm A = \bm G\bm G^{\top}$.
\end{enumerate}
\end{theorembox}

\subsubsection{Characterizations of Positive Definiteness}

\begin{theorembox}
For $\bm A\in \Sbb^n$, the following are equivalent:
\begin{enumerate}
  \item $\bm A$ is positive definite.
  \item All eigenvalues of $\bm A$ are $>0$.
  \item There exists a matrix $\bm G$ such that $\bm A = \bm G\bm G^{\top}$ and $\operatorname{rank}(\bm G) = n$.
\end{enumerate}
\end{theorembox}

\medskip

\begin{factbox}[title={}]
\textbf{Notation.}
\begin{itemize}
  \item $\bm A \succeq \bm B$ means $\bm A - \bm B$ is positive semidefinite.
  \item $\bm A \succ \bm B$ means $\bm A - \bm B$ is positive definite.
  \item $\Spp$ denotes the set of positive semidefinite matrices.
  \item $\mathbb S_{++}^n$ denotes the set of positive definite matrices.
\end{itemize}
\end{factbox}

\subsection{Topology Review}

Point-set topology of $\R^n$.

The \textbf{open ball} of radius $r>0$ around $\bm x\in \R^n$ is
\[
\mathbb B(\bm x,r) = \{\bm y\in \R^n : \|\bm y - \bm x\| < r\}.
\]

The \textbf{closed ball} of radius $r>0$ around $\bm x\in \R^n$ is
\[
\overline{\mathbb B}(\bm x,r) = \{\bm y\in \R^n : \|\bm y - \bm x\| \le r\}.
\]

\begin{factbox}
\begin{itemize}
  \item A set $S\subseteq \R^n$ is \textbf{open} if for every $\bm x\in S$ there exists $r>0$ such that
        \[
        \mathbb B(\bm x,r) \subseteq S.
        \]
  \item A set $S\subseteq \R^n$ is \textbf{closed} if every convergent sequence in $S$ has its limit in $S$.
\end{itemize}
\end{factbox}

A set that contains only part of its boundary is, in general, neither open nor closed.
