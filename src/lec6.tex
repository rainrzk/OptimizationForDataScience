\section{Lec 6}

\subsection{Convex Functions}

\subsubsection{Definitions}

Let $f:\Omega\to \R$, $\Omega\subseteq \R^n$, and $\Omega = \operatorname{dom}(f)$. Assume $\Omega$ is convex.  
We say $f$ is a \textbf{convex function} if
\[
\forall \bm x,\bm y\in\Omega,\ \forall \lambda\in[0,1],\quad
f\big((1-\lambda)\bm x+\lambda \bm y\big)
\le (1-\lambda)f(\bm x) + \lambda f(\bm y).
\]

\paragraph{Extended-value notation.}
If $f:\Omega\to \R$ is convex, we extend it by setting
\[
f(\bm x) = \infty \quad \text{for } \bm x\notin \Omega.
\]
Thus we can regard
\[
f:\R^n\to \R\cup\{\infty\},
\]
and define
\[
\operatorname{dom}(f) = \{\bm x\in \R^n : f(\bm x)<\infty\}.
\]
The convexity inequality above is then interpreted for all $\bm x,\bm y\in \R^n$; it enforces that $\operatorname{dom}(f)$ is convex.

\subsubsection{Examples}

\begin{examplebox}
\[
\begin{aligned}
1.\quad &f(\bm x)
= \frac12 \bm x^{\top} \bm H\bm x
  + \bm g^{\top}\bm x + d,
\quad  \bm H\succeq \0.\\[0.3em]
2.\quad &f(\bm x) = \|\bm x\|_\square
\quad\text{(any norm)}.\\[0.3em]
3.\quad &f(x) = \mathrm e^x.\\[0.3em]
4.\quad &f(x) =
\begin{cases}
-\ln x, & x>0,\\
\infty, & x\le 0,
\end{cases}\\[0.5em]
5.\quad &f(x) =
\begin{cases}
\dfrac1x, & x>0,\\[0.2em]
\infty, & x\le 0,
\end{cases}\\[0.5em]
6.\quad &\text{If } g_1,\dots,g_m:\R^n\to\R\cup\{\infty\}
\text{ are convex and } \alpha_1,\dots,\alpha_m\ge 0,\\
&\quad\text{then } \alpha_1 g_1 + \cdots + \alpha_m g_m \text{ is convex}.\\[0.3em]
7.\quad &\text{Under the same assumption as in 6, }
\max\{\alpha_1 g_1,\dots,\alpha_m g_m\} \text{ is convex.}\\
&\quad\text{Example: } f(x) = \max(x,-x) = |x|\text{ is convex.}
\end{aligned}
\]
\end{examplebox}

\begin{theorembox}
If $f:\R^n\to \R\cup\{\infty\}$ is convex and $\bm x^*\in \operatorname{dom}(f)$ is a local minimizer, then $\bm x^*$ is a global minimizer.
\end{theorembox}

\begin{proofbox}
Assume $\bm x^*$ is a local minimizer and let $\bm y\in \R^n$ be arbitrary. Define
\[
\bm y_i := \left(1-\frac1i\right)\bm x^* + \frac1i\,\bm y,
\quad i=1,2,3,\dots
\]
Then $\bm y_i \to \bm x^*$ as $i\to\infty$. Since $\bm x^*$ is a local minimizer, there exists $k$ such that
\[
f(\bm y_i) \ge f(\bm x^*),\quad \forall i\ge k.
\]
In particular,
\[
f\!\left(\left(1-\frac1k\right)\bm x^* + \frac1k\,\bm y\right)
\ge f(\bm x^*).
\]

By convexity,
\[
f\!\left(\left(1-\frac1k\right)\bm x^* + \frac1k\,\bm y\right)
\le \left(1-\frac1k\right)f(\bm x^*) + \frac1k f(\bm y).
\]
Combining,
\[
\left(1-\frac1k\right)f(\bm x^*) + \frac1k f(\bm y)
\ge f(\bm x^*).
\]
Rearranging gives
\[
f(\bm y) \ge f(\bm x^*).
\]
Since $\bm y$ was arbitrary, $\bm x^*$ is a global minimizer.
\end{proofbox}

\subsection{Differentiable Convex Functions}

Suppose $f:\R^n\to \R\cup\{\infty\}$ is convex and differentiable at $\bm x\in \intdom(f)$. Then for all $\bm y\in \R^n$ and all $\alpha\in[0,1]$,
\[
f((1-\alpha)\bm x + \alpha \bm y)
\le (1-\alpha)f(\bm x) + \alpha f(\bm y).
\]

Write the left-hand side using differentiability at $\bm x$:
\[
\begin{aligned}
f((1-\alpha)\bm x + \alpha \bm y)
&= f(\bm x + \alpha(\bm y-\bm x))\\
&= f(\bm x)
 + \alpha\,\nabla f(\bm x)^{\top}(\bm y-\bm x)
 + \phi(\alpha(\bm y-\bm x)),
\end{aligned}
\]
for some remainder term $\phi$ with
\[
\lim_{\bm h\to \bm0,\ \bm h\ne \bm0}
\frac{\phi(\bm h)}{\|\bm h\|} = 0.
\]

Combining with convexity,
\[
f(\bm x)
+ \alpha\,\nabla f(\bm x)^{\top}(\bm y-\bm x)
+ \phi(\alpha(\bm y-\bm x))
\le (1-\alpha)f(\bm x) + \alpha f(\bm y).
\]
Assuming $\alpha>0$, subtract $f(\bm x)$ and divide by $\alpha$:
\[
\nabla f(\bm x)^{\top}(\bm y-\bm x)
+ \frac{\phi(\alpha(\bm y-\bm x))}{\alpha}
\le f(\bm y) - f(\bm x).
\]
Taking the limit $\alpha\downarrow 0$, the remainder term vanishes and we obtain the

\begin{factbox}
\textbf{Subgradient Inequality.} For convex differentiable $f$ and any $\bm x\in \intdom(f)$,
\[
f(\bm y)
\ge f(\bm x) + \nabla f(\bm x)^{\top}(\bm y-\bm x),
\quad \forall \bm y\in \R^n.
\]
\end{factbox}

The converse also holds: if this inequality is satisfied for all $\bm x,\bm y$, then $f$ is convex.

\begin{theorembox}
Let $f:\R^n\to \R\cup\{\infty\}$ be convex. Suppose $\bm x^*\in \intdom(f)$, $f$ is differentiable at $\bm x^*$, and $\nabla f(\bm x^*) = \0$. Then $\bm x^*$ is a global minimizer of $f$.
\end{theorembox}

\begin{proofbox}
Applying the subgradient inequality at $\bm x^*$,
\[
f(\bm y)
\ge f(\bm x^*) + \nabla f(\bm x^*)^{\top}(\bm y-\bm x^*)
= f(\bm x^*),
\quad \forall \bm y\in \R^n,
\]
since $\nabla f(\bm x^*)=\0$. Hence $\bm x^*$ is a global minimizer.
\end{proofbox}

\subsection{Descent Methods}

\subsubsection{Descent Directions}

Let $f:\Omega\to \R$, $\Omega\subseteq \R^n$, and $\bm x\in\Omega$.  
We say $\bm d\in \R^n$ is a \textbf{descent direction} at $\bm x$ if there exists $\bar t>0$ such that
\[
\bm x + t\bm d \in \Omega,\ \forall t\in[0,\bar t],
\quad \text{and}\quad
f(\bm x + t\bm d) < f(\bm x),
\ \forall t\in(0,\bar t].
\]

\begin{theorembox}
Let $f:\Omega\to \R$ with $\Omega\subseteq \R^n$. Assume $\bm x\in \operatorname{int}(\Omega)$ and $f$ is differentiable at $\bm x$. Then any $\bm d$ satisfying
\[
\bm d^{\top}\nabla f(\bm x) < 0
\]
is a descent direction at $\bm x$.
\end{theorembox}

\begin{proofbox}
Assume $\nabla f(\bm x)\neq \0$ and $\bm d\neq \0$ (otherwise the statement is vacuous). Differentiability at $\bm x$ means
\[
\lim_{\bm h\to \bm0,\ \bm h\ne \bm0}
\frac{f(\bm x+\bm h) - f(\bm x)
      - \nabla f(\bm x)^{\top}\bm h}
     {\|\bm h\|}
= 0.
\]
Take $\bm h = t\bm d$ with $t\downarrow 0$:
\[
\lim_{t\downarrow 0}
\frac{f(\bm x+t\bm d) - f(\bm x)
      - t\,\nabla f(\bm x)^{\top}\bm d}{t}
= 0,
\]
or equivalently,
\[
\lim_{t\downarrow 0}
\left(
\frac{f(\bm x+t\bm d) - f(\bm x)}{t}
- \nabla f(\bm x)^{\top}\bm d
\right) = 0.
\]

By assumption, $\nabla f(\bm x)^{\top}\bm d < 0$. Therefore, for sufficiently small $t>0$,
\[
\frac{f(\bm x+t\bm d) - f(\bm x)}{t} < 0,
\]
which implies $f(\bm x+t\bm d) < f(\bm x)$ for all sufficiently small $t>0$. Thus $\bm d$ is a descent direction.
\end{proofbox}

\paragraph{Steepest descent direction.}
Assume $\nabla f(\bm x)\neq \0$. Among all unit vectors, the direction that minimizes the directional derivative is
\[
\arg\min\{\bm d^{\top}\nabla f(\bm x) : \|\bm d\|=1\}
= -\frac{\nabla f(\bm x)}{\|\nabla f(\bm x)\|}.
\]

This follows from the strong form of the Cauchy–Schwarz inequality:
\[
\bm u^{\top}\bm v \le \|\bm u\|\,\|\bm v\|
\]
with equality if and only if either $\bm u = \0$ or $\bm v = \0$, or there exists $\lambda\ge 0$ such that $\bm u = \lambda \bm v$.

\subsubsection{Gradient Descent Algorithm (Steepest Descent)}

Let $f:\R^n\to \R$ be differentiable, and choose an initial point $\bm x^0\in \R^n$.

\medskip
For $k=0,1,2,\dots$:
\begin{itemize}
  \item select a step size $\alpha_k>0$,
  \item update
  \[
  \bm x^{k+1} := \bm x^k - \alpha_k \nabla f(\bm x^k).
  \]
\end{itemize}
End.

\medskip
Here $\alpha_k$ is the \textbf{step size} (learning rate). The procedure used to choose $\alpha_k$ is called a \textbf{line search}.

If $f$ is $L$-smooth, then the constant choice $\alpha_k\equiv \dfrac1L$ always works.  
Any fixed $\alpha_k\in (0,\dfrac1L)$ also leads to convergence but typically more slowly.  
A common stopping rule is $\|\nabla f(\bm x^k)\|\le \text{tolerance}$.

\begin{theorembox}
(\textbf{Convex case of Gradient Descent})  
Let $f:\R^n\to \R$ be convex and $L$-smooth. Suppose there exists $\bm x^*$ such that
\[
\nabla f(\bm x^*) = \0.
\]
Run gradient descent from $\bm x^0$ with step sizes $\alpha_k \equiv \dfrac1L$. Then
\[
f(\bm x^k) - f(\bm x^*)
\le \frac{L\|\bm x^0 - \bm x^*\|^2}{2k},
\quad \forall k=1,2,\dots
\]
\end{theorembox}

\begin{proofbox}
From $L$-smoothness,
\[
\begin{aligned}
f(\bm x^{k+1})
&\le f(\bm x^k)
  + \nabla f(\bm x^k)^{\top}(\bm x^{k+1}-\bm x^k)
  + \frac{L}{2}\|\bm x^{k+1}-\bm x^k\|^2\\[0.3em]
&= f(\bm x^k)
  + \nabla f(\bm x^k)^{\top}\Bigl(-\frac1L\nabla f(\bm x^k)\Bigr)
  + \frac{L}{2}\Bigl\|-\frac1L\nabla f(\bm x^k)\Bigr\|^2\\[0.3em]
&= f(\bm x^k)
  - \frac1L\|\nabla f(\bm x^k)\|^2
  + \frac1{2L}\|\nabla f(\bm x^k)\|^2\\[0.3em]
&= f(\bm x^k) - \frac1{2L}\|\nabla f(\bm x^k)\|^2.
\end{aligned}
\]

From the subgradient inequality for convex differentiable $f$,
\[
f(\bm x^*)
\ge f(\bm x^k)
 + \nabla f(\bm x^k)^{\top}(\bm x^*-\bm x^k),
\]
so
\[
f(\bm x^k)
\le f(\bm x^*)
 + \nabla f(\bm x^k)^{\top}(\bm x^k-\bm x^*).
\]

Combining,
\[
f(\bm x^{k+1})
\le f(\bm x^*)
 + \nabla f(\bm x^k)^{\top}(\bm x^k-\bm x^*)
 - \frac1{2L}\|\nabla f(\bm x^k)\|^2.
\]

Using the update
\(
\bm x^{k+1}
= \bm x^k - \dfrac1L\nabla f(\bm x^k),
\)
we compute
\[
\begin{aligned}
\|\bm x^{k+1}-\bm x^*\|^2
&= \left\|\bm x^k - \frac1L\nabla f(\bm x^k) - \bm x^*\right\|^2\\
&= \|\bm x^k-\bm x^*\|^2
  - \frac2L\,\nabla f(\bm x^k)^{\top}(\bm x^k-\bm x^*)
  + \frac1{L^2}\|\nabla f(\bm x^k)\|^2.
\end{aligned}
\]

Rearranging,
\[
\nabla f(\bm x^k)^{\top}(\bm x^k-\bm x^*)
= \frac{L}{2}\bigl(\|\bm x^k-\bm x^*\|^2
                 - \|\bm x^{k+1}-\bm x^*\|^2\bigr)
  + \frac1{2L}\|\nabla f(\bm x^k)\|^2.
\]

Substitute into the bound for $f(\bm x^{k+1})$:
\[
\begin{aligned}
f(\bm x^{k+1})
&\le f(\bm x^*)
 + \frac{L}{2}\bigl(\|\bm x^k-\bm x^*\|^2
                 - \|\bm x^{k+1}-\bm x^*\|^2\bigr)
 + \frac1{2L}\|\nabla f(\bm x^k)\|^2
 - \frac1{2L}\|\nabla f(\bm x^k)\|^2\\[0.3em]
&= f(\bm x^*)
 + \frac{L}{2}\bigl(\|\bm x^k-\bm x^*\|^2
                 - \|\bm x^{k+1}-\bm x^*\|^2\bigr).
\end{aligned}
\]

Thus
\[
f(\bm x^{k+1}) - f(\bm x^*)
\le \frac{L}{2}\bigl(\|\bm x^k-\bm x^*\|^2
                 - \|\bm x^{k+1}-\bm x^*\|^2\bigr).
\]

Summing from $k=0$ to $k=\ell-1$,
\[
\sum_{k=0}^{\ell-1}\bigl(f(\bm x^{k+1}) - f(\bm x^*)\bigr)
\le \frac{L}{2}\bigl(\|\bm x^0-\bm x^*\|^2
                 - \|\bm x^\ell-\bm x^*\|^2\bigr)
\le \frac{L}{2}\|\bm x^0-\bm x^*\|^2.
\]

Since $f(\bm x^{k+1}) - f(\bm x^*)\ge 0$ and is nonincreasing, we have
\[
\sum_{k=0}^{\ell-1}\bigl(f(\bm x^{k+1}) - f(\bm x^*)\bigr)
\ge \ell\bigl(f(\bm x^\ell) - f(\bm x^*)\bigr).
\]

Therefore,
\[
\ell\bigl(f(\bm x^\ell) - f(\bm x^*)\bigr)
\le \frac{L}{2}\|\bm x^0-\bm x^*\|^2,
\]
which gives
\[
f(\bm x^\ell) - f(\bm x^*)
\le \frac{L\|\bm x^0-\bm x^*\|^2}{2\ell}.
\]

Renaming $\ell$ as $k$ yields the stated result.
\end{proofbox}
