\section{Lec 16}

\subsection{Composite Minimization}

We consider the problem
\[
\min_{\bm x} F(\bm x),
\qquad
F(\bm x) \defeq f(\bm x) + \psi(\bm x),
\]
where
\begin{itemize}
  \item $f:\R^n\to\R$ is convex and $L$-smooth;
  \item $\psi:\R^n\to\R\cup\{\infty\}$ is proper, closed, and convex.
\end{itemize}

\begin{examplebox}
\begin{itemize}
  \item SVM-8 (hinge loss + quadratic regularizer).
  \item $\ell_1$-regularized least squares:
  \[
    \ell_1\text{LS}:\quad
    \min_{\bm x}\;
    \underbrace{\frac12\|\bm A\bm x-\bm b\|^2}_{f(\bm x)}
    + \underbrace{\gamma\|\bm x\|_1}_{\psi(\bm x)}.
  \]
\end{itemize}
\end{examplebox}

\begin{theorembox}
\textbf{Optimality condition for composite minimization.}

For $F(\bm x)=f(\bm x)+\psi(\bm x)$ as above,
\[
\bm x\in\argmin F
\quad\Longleftrightarrow\quad
-\nabla f(\bm x)\in\partial\psi(\bm x).
\]
\end{theorembox}

\begin{proofbox}
Since $F=f+\psi$ with $f$ differentiable,
\[
\partial F(\bm x) = \partial f(\bm x) + \partial\psi(\bm x)
= \{\nabla f(\bm x)\} + \partial\psi(\bm x).
\]
By the subgradient optimality condition,
\[
\bm x\in\argmin F
\quad\Longleftrightarrow\quad
\bm 0\in\partial F(\bm x)
\quad\Longleftrightarrow\quad
\bm 0\in \{\nabla f(\bm x)\}+\partial\psi(\bm x),
\]
which is equivalent to $-\nabla f(\bm x)\in\partial\psi(\bm x)$.
\end{proofbox}

\begin{theorembox}
\textbf{Uniqueness under strong convexity.}

Suppose, in addition, that $f$ is $m$-strongly convex with $m>0$. Then
$F=f+\psi$ has a unique minimizer.
\end{theorembox}

\begin{proofbox}
A sum of a strongly convex function and a convex function is strongly
convex. Thus $F$ is strongly convex, so it is strictly convex and hence
has at most one minimizer. Since $F$ is proper, closed, and coercive
(cf.\ the level-set argument below), a minimizer exists and must be
unique.
\end{proofbox}

\begin{claimbox}
For any $\bm x_0\in\operatorname{dom}(\psi)$, the level set
\[
S \defeq \{\bm x\in\R^n : F(\bm x)\le F(\bm x_0)\}
\]
is closed.
\end{claimbox}

\begin{proofbox}
Both $f$ and $\psi$ are closed, so their sum $F=f+\psi$ is closed.
For a closed function, all sublevel sets $\{\bm x : F(\bm x)\le\beta\}$
are closed. Taking $\beta=F(\bm x_0)$ gives that $S$ is closed.
\end{proofbox}

\begin{claimbox}
Choose $\bm x_0\in\operatorname{ri}(\operatorname{dom}(\psi))$ and let
$S$ be the level set above. Then $S$ is bounded.
\end{claimbox}

\begin{proofbox}
Because $\bm x_0\in\operatorname{ri}(\operatorname{dom}(\psi))$ and
$\psi$ is convex, we have $\partial\psi(\bm x_0)\ne\varnothing$.
Choose some $\bm g\in\partial\psi(\bm x_0)$. Then for all $\bm x$,
\[
\psi(\bm x) \ge \psi(\bm x_0) + \bm g^\top(\bm x-\bm x_0).
\]
Since $f$ is $m$-strongly convex,
\[
f(\bm x) \ge f(\bm x_0)
+ \nabla f(\bm x_0)^\top(\bm x-\bm x_0)
+ \frac{m}{2}\|\bm x-\bm x_0\|^2.
\]
Adding gives
\[
F(\bm x)
\ge F(\bm x_0)
+ (\nabla f(\bm x_0)+\bm g)^\top(\bm x-\bm x_0)
+ \frac{m}{2}\|\bm x-\bm x_0\|^2.
\]

For $\bm x\in S$, we have $F(\bm x)-F(\bm x_0)\le 0$, hence
\[
0 \ge F(\bm x)-F(\bm x_0)
\ge (\nabla f(\bm x_0)+\bm g)^\top(\bm x-\bm x_0)
  + \frac{m}{2}\|\bm x-\bm x_0\|^2.
\]
Applying Cauchy–Schwarz,
\[
(\nabla f(\bm x_0)+\bm g)^\top(\bm x-\bm x_0)
\ge -\|\nabla f(\bm x_0)+\bm g\|\;\|\bm x-\bm x_0\|.
\]
Thus
\[
0
\ge -\|\nabla f(\bm x_0)+\bm g\|\;\|\bm x-\bm x_0\|
  + \frac{m}{2}\|\bm x-\bm x_0\|^2.
\]
Rearranging (and assuming $\bm x\ne\bm x_0$ so the norm is nonzero) gives
\[
\frac{m}{2}\,\|\bm x-\bm x_0\|
\le \|\nabla f(\bm x_0)+\bm g\|.
\]
Hence $\|\bm x-\bm x_0\|$ is uniformly bounded for all $\bm x\in S$.
Therefore $S$ is bounded.
\end{proofbox}

\begin{claimbox}
$F$ is bounded below on $S$ and attains a minimum over $S$ (and hence
over $\R^n$).
\end{claimbox}

\begin{proofbox}
For any $\bm x\in S$, using the same inequality as above but dropping
the quadratic term yields
\[
F(\bm x)
\ge F(\bm x_0)
+ (\nabla f(\bm x_0)+\bm g)^\top(\bm x-\bm x_0)
\ge F(\bm x_0)
- \|\nabla f(\bm x_0)+\bm g\|\;\|\bm x-\bm x_0\|.
\]
Using the bound on $\|\bm x-\bm x_0\|$ from Claim~2,
we obtain a uniform lower bound:
\[
F(\bm x) \ge F_{\min} \quad\text{for all } \bm x\in S
\]
for some finite $F_{\min}$.

Since $S$ is closed and bounded, it is compact. The epigraph
\[
\operatorname{epi}(F)
= \{(\bm x,y) : y\ge F(\bm x)\}
\]
is closed, so
\[
\operatorname{epi}(F) \cap (S\times[F_{\min},\infty))
\]
is a nonempty compact set. The projection map
$\phi(\bm x,y)=y$ attains its minimum on this set at some
$(\bm x^*,F(\bm x^*))$. Thus $F$ attains its minimum on $S$ at
$\bm x^*$.

For any $\bm x\notin S$, we have $F(\bm x)>F(\bm x_0)\ge F(\bm x^*)$,
so $\bm x^*$ is in fact a global minimizer of $F$ over $\R^n$.
\end{proofbox}

\subsection{Proximal Operator}

Let $\psi:\R^n\to\R\cup\{\infty\}$ be proper, closed, and convex.
The \textbf{proximal operator} of $\psi$ is defined by
\[
\operatorname{prox}_{\psi}(\bm x)
\defeq
\argmin_{\bm z}\Bigl\{\psi(\bm z)+\frac12\|\bm z-\bm x\|^2\Bigr\}.
\]

Intuitively, $\operatorname{prox}_{\psi}(\bm x)$ finds a point $\bm z$
that balances two goals:
staying close to the current point $\bm x$, and having a small value of
the regularizer $\psi$.

\begin{theorembox}
\textbf{Existence and uniqueness of the proximal point.}

For each $\bm x\in\R^n$, the function
\[
\bm z\mapsto \psi(\bm z)+\frac12\|\bm z-\bm x\|^2
\]
has a unique minimizer, i.e.\ $\operatorname{prox}_{\psi}(\bm x)$ is
well-defined and single-valued.
\end{theorembox}

\begin{proofbox}
The function $\bm z\mapsto \frac12\|\bm z-\bm x\|^2$ is strongly convex.
Adding the convex function $\psi$ preserves strong convexity, so
$\psi(\bm z)+\frac12\|\bm z-\bm x\|^2$ is strongly convex and proper,
closed. Hence it has a unique minimizer.
\end{proofbox}

We say that $\psi$ is \emph{proximable} if there exists an efficient
algorithm to compute $\operatorname{prox}_{\psi}(\bm x)$.

\begin{examplebox}
\textbf{Example 1: Scalar $\ell_1$-prox.}

Let $n=1$ and
\[
\psi(x) = t|x|,\quad t>0.
\]
Define
\[
q(z) \defeq t|z|+\frac12(z-x)^2.
\]
Then $\operatorname{prox}_{t|\cdot|}(x)$ is the unique minimizer of $q$.
We have
\[
\partial q(z) = t\,\partial|z| + (z-x).
\]
Recall
\[
\partial|z|=
\begin{cases}
\{-1\}, & z<0,\\[0.2em]
[-1,1], & z=0,\\[0.2em]
\{1\},  & z>0.
\end{cases}
\]

The optimality condition
\[
0\in\partial q(z)
\quad\Longleftrightarrow\quad
0\in t\,\partial|z| + z - x
\]
leads to three cases:
\[
x-z\in
\begin{cases}
\{-t\}, & z<0,\\
[-t,t], & z=0,\\
\{t\},  & z>0.
\end{cases}
\]

\begin{itemize}
  \item \textbf{Case $z<0$:} $x-z=-t \Rightarrow z=x+t$. Validity
        requires $z<0\Rightarrow x<-t$.
  \item \textbf{Case $z=0$:} $x-z\in[-t,t] \Rightarrow x\in[-t,t]$.
  \item \textbf{Case $z>0$:} $x-z=t \Rightarrow z=x-t$. Validity
        requires $z>0\Rightarrow x>t$.
\end{itemize}

Summarizing,
\[
\operatorname{prox}_{t|\cdot|}(x)
=
\begin{cases}
x+t, & x<-t,\\[0.2em]
0,   & x\in[-t,t],\\[0.2em]
x-t, & x>t.
\end{cases}
\]
This is the usual scalar \emph{soft-thresholding} operator.
\end{examplebox}

\begin{examplebox}
\textbf{Example 2: Vector $\ell_1$-prox.}

For $\bm x\in\R^n$ and $t>0$,
\[
\operatorname{prox}_{t\|\cdot\|_1}(\bm x)
= \argmin_{\bm z} \bigl\{t\|\bm z\|_1 + \tfrac12\|\bm z-\bm x\|^2\bigr\}.
\]
Because the function is separable across coordinates, the solution is
obtained by applying the scalar soft thresholding coordinatewise:
\[
\operatorname{prox}_{t\|\cdot\|_1}(\bm x)
= \begin{pmatrix} u_1 \\ \vdots \\ u_n \end{pmatrix}
\quad\text{with}\quad
u_i =
\begin{cases}
x_i+t, & x_i<-t,\\[0.2em]
0,     & x_i\in[-t,t],\\[0.2em]
x_i-t, & x_i>t,
\end{cases}
\quad i=1,\dots,n.
\]
\end{examplebox}

\begin{examplebox}
\textbf{Example 3: Prox of the Euclidean norm.}

Let
\[
\psi(\bm x) = t\|\bm x\|,\quad t>0.
\]
We know
\[
\partial\|\bm z\|=
\begin{cases}
\left\{\dfrac{\bm z}{\|\bm z\|}\right\}, & \bm z\ne\bm 0,\\[0.4em]
\overline{\mathbb B}(\bm 0,1), & \bm z=\bm 0,
\end{cases}
\]
where $\overline{\mathbb B}(\bm 0,1)$ is the closed unit ball.

Consider
\[
q(\bm z) \defeq t\|\bm z\| + \frac12\|\bm z-\bm x\|^2.
\]
Then
\[
\bm 0\in\partial q(\bm z)
\quad\Longleftrightarrow\quad
\bm 0\in t\,\partial\|\bm z\| + (\bm z-\bm x),
\]
i.e.
\[
\bm x-\bm z \in
\begin{cases}
\overline{\mathbb B}(\bm 0,t), & \bm z=\bm 0 \quad (\text{a}),\\[0.2em]
t\,\dfrac{\bm z}{\|\bm z\|},   & \bm z\ne\bm 0 \quad (\text{b}).
\end{cases}
\]

\textbf{Case (a):} $\bm z=\bm 0$ is valid if and only if
$\bm x\in\overline{\mathbb B}(\bm 0,t)$, i.e.\ $\|\bm x\|\le t$.

\textbf{Case (b):} Assume $\bm z\ne\bm 0$ and write $\bm z=\theta\bm x$
for some scalar $\theta$. Then
\[
\bm x-\theta\bm x
= t\,\frac{\theta\bm x}{\|\theta\bm x\|}
= t\,\frac{\theta}{|\theta|}\,\frac{\bm x}{\|\bm x\|}.
\]
Assuming $\theta>0$ (we expect $\bm z$ to be in the same direction as
$\bm x$), we get
\[
1-\theta = \frac{t}{\|\bm x\|}\quad\Rightarrow\quad
\theta = 1 - \frac{t}{\|\bm x\|}.
\]
Thus
\[
\bm z
= \left(1-\frac{t}{\|\bm x\|}\right)\bm x,
\]
which is valid only when $\bm z\ne\bm 0$, i.e.\ $\|\bm x\|>t$.

Combining both cases:
\[
\operatorname{prox}_{t\|\cdot\|}(\bm x)
=
\begin{cases}
\bm 0, & \|\bm x\|\le t,\\[0.3em]
\left(1-\dfrac{t}{\|\bm x\|}\right)\bm x,
& \|\bm x\|>t.
\end{cases}
\]
This is often called the \emph{vector soft-thresholding} or
\emph{shrinkage} operator.
\end{examplebox}

\begin{factbox}
For any closed, proper, convex function $\psi$, the proximal map
$\operatorname{prox}_{\psi}:\R^n\to\R^n$ is 1-Lipschitz, i.e.
\[
\|\operatorname{prox}_{\psi}(\bm x)
 - \operatorname{prox}_{\psi}(\bm y)\|
\le \|\bm x-\bm y\|,\quad \forall\,\bm x,\bm y\in\R^n.
\]
In particular, the proximal map is continuous.
\end{factbox}

\subsection{Proximal Gradient}

We return to the composite setup
\[
\min_{\bm x} F(\bm x),
\qquad
F(\bm x)=f(\bm x)+\psi(\bm x),
\]
with $f$ convex and $L$-smooth and $\psi$ proper, closed, convex.

\begin{factbox}[title={}]
\textbf{Proximal gradient.}

The \emph{proximal gradient} of $F$ (with parameter $L$) is
\[
\text G_L(\bm x)
\defeq
L\left(
\bm x
- \operatorname{prox}_{\psi/L}\!\left(
\bm x - \frac1L\nabla f(\bm x)
\right)
\right).
\]
\end{factbox}

Intuitively, the mapping
\[
\bm x \mapsto \operatorname{prox}_{\psi/L}\!\left(
\bm x - \frac1L\nabla f(\bm x)
\right)
\]
performs one gradient step on the smooth part $f$ followed by one prox
step on the (possibly nonsmooth) part $\psi$. The vector $\text G_L(\bm x)$ is
the scaled difference between the current point and this update, and
plays the role of a generalized gradient.

\begin{examplebox}
\textbf{Special case $\psi\equiv 0$.}

If $\psi\equiv 0$, then
\[
\operatorname{prox}_0(\bm x)
= \argmin_{\bm z}\frac12\|\bm z-\bm x\|^2
= \bm x.
\]
Hence
\[
\text G_L(\bm x)
= L\left(
\bm x - \left(\bm x - \frac1L\nabla f(\bm x)\right)
\right)
= \nabla f(\bm x),
\]
so the proximal gradient reduces to the usual gradient.
\end{examplebox}
